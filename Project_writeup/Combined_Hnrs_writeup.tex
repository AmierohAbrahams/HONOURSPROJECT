\documentclass[12pt,a4paper,]{article}
\usepackage{lmodern}
\usepackage{amssymb,amsmath}
\usepackage{ifxetex,ifluatex}
\usepackage{fixltx2e} % provides \textsubscript
\ifnum 0\ifxetex 1\fi\ifluatex 1\fi=0 % if pdftex
  \usepackage[T1]{fontenc}
  \usepackage[utf8]{inputenc}
\else % if luatex or xelatex
  \ifxetex
    \usepackage{mathspec}
  \else
    \usepackage{fontspec}
  \fi
  \defaultfontfeatures{Ligatures=TeX,Scale=MatchLowercase}
    \setmainfont[]{PT Serif}
    \setsansfont[]{PT Sans}
    \setmonofont[Mapping=tex-ansi]{PT Mono}
\fi
% use upquote if available, for straight quotes in verbatim environments
\IfFileExists{upquote.sty}{\usepackage{upquote}}{}
% use microtype if available
\IfFileExists{microtype.sty}{%
\usepackage{microtype}
\UseMicrotypeSet[protrusion]{basicmath} % disable protrusion for tt fonts
}{}
\usepackage[margin=1in]{geometry}
\usepackage{hyperref}
\hypersetup{unicode=true,
            pdftitle={Complete},
            pdfauthor={Amieroh Abrahams},
            pdfborder={0 0 0},
            breaklinks=true}
\urlstyle{same}  % don't use monospace font for urls
\usepackage{color}
\usepackage{fancyvrb}
\newcommand{\VerbBar}{|}
\newcommand{\VERB}{\Verb[commandchars=\\\{\}]}
\DefineVerbatimEnvironment{Highlighting}{Verbatim}{commandchars=\\\{\}}
% Add ',fontsize=\small' for more characters per line
\usepackage{framed}
\definecolor{shadecolor}{RGB}{248,248,248}
\newenvironment{Shaded}{\begin{snugshade}}{\end{snugshade}}
\newcommand{\AlertTok}[1]{\textcolor[rgb]{0.94,0.16,0.16}{#1}}
\newcommand{\AnnotationTok}[1]{\textcolor[rgb]{0.56,0.35,0.01}{\textbf{\textit{#1}}}}
\newcommand{\AttributeTok}[1]{\textcolor[rgb]{0.77,0.63,0.00}{#1}}
\newcommand{\BaseNTok}[1]{\textcolor[rgb]{0.00,0.00,0.81}{#1}}
\newcommand{\BuiltInTok}[1]{#1}
\newcommand{\CharTok}[1]{\textcolor[rgb]{0.31,0.60,0.02}{#1}}
\newcommand{\CommentTok}[1]{\textcolor[rgb]{0.56,0.35,0.01}{\textit{#1}}}
\newcommand{\CommentVarTok}[1]{\textcolor[rgb]{0.56,0.35,0.01}{\textbf{\textit{#1}}}}
\newcommand{\ConstantTok}[1]{\textcolor[rgb]{0.00,0.00,0.00}{#1}}
\newcommand{\ControlFlowTok}[1]{\textcolor[rgb]{0.13,0.29,0.53}{\textbf{#1}}}
\newcommand{\DataTypeTok}[1]{\textcolor[rgb]{0.13,0.29,0.53}{#1}}
\newcommand{\DecValTok}[1]{\textcolor[rgb]{0.00,0.00,0.81}{#1}}
\newcommand{\DocumentationTok}[1]{\textcolor[rgb]{0.56,0.35,0.01}{\textbf{\textit{#1}}}}
\newcommand{\ErrorTok}[1]{\textcolor[rgb]{0.64,0.00,0.00}{\textbf{#1}}}
\newcommand{\ExtensionTok}[1]{#1}
\newcommand{\FloatTok}[1]{\textcolor[rgb]{0.00,0.00,0.81}{#1}}
\newcommand{\FunctionTok}[1]{\textcolor[rgb]{0.00,0.00,0.00}{#1}}
\newcommand{\ImportTok}[1]{#1}
\newcommand{\InformationTok}[1]{\textcolor[rgb]{0.56,0.35,0.01}{\textbf{\textit{#1}}}}
\newcommand{\KeywordTok}[1]{\textcolor[rgb]{0.13,0.29,0.53}{\textbf{#1}}}
\newcommand{\NormalTok}[1]{#1}
\newcommand{\OperatorTok}[1]{\textcolor[rgb]{0.81,0.36,0.00}{\textbf{#1}}}
\newcommand{\OtherTok}[1]{\textcolor[rgb]{0.56,0.35,0.01}{#1}}
\newcommand{\PreprocessorTok}[1]{\textcolor[rgb]{0.56,0.35,0.01}{\textit{#1}}}
\newcommand{\RegionMarkerTok}[1]{#1}
\newcommand{\SpecialCharTok}[1]{\textcolor[rgb]{0.00,0.00,0.00}{#1}}
\newcommand{\SpecialStringTok}[1]{\textcolor[rgb]{0.31,0.60,0.02}{#1}}
\newcommand{\StringTok}[1]{\textcolor[rgb]{0.31,0.60,0.02}{#1}}
\newcommand{\VariableTok}[1]{\textcolor[rgb]{0.00,0.00,0.00}{#1}}
\newcommand{\VerbatimStringTok}[1]{\textcolor[rgb]{0.31,0.60,0.02}{#1}}
\newcommand{\WarningTok}[1]{\textcolor[rgb]{0.56,0.35,0.01}{\textbf{\textit{#1}}}}
\usepackage{longtable,booktabs}
\usepackage{graphicx,grffile}
\makeatletter
\def\maxwidth{\ifdim\Gin@nat@width>\linewidth\linewidth\else\Gin@nat@width\fi}
\def\maxheight{\ifdim\Gin@nat@height>\textheight\textheight\else\Gin@nat@height\fi}
\makeatother
% Scale images if necessary, so that they will not overflow the page
% margins by default, and it is still possible to overwrite the defaults
% using explicit options in \includegraphics[width, height, ...]{}
\setkeys{Gin}{width=\maxwidth,height=\maxheight,keepaspectratio}
\IfFileExists{parskip.sty}{%
\usepackage{parskip}
}{% else
\setlength{\parindent}{0pt}
\setlength{\parskip}{6pt plus 2pt minus 1pt}
}
\setlength{\emergencystretch}{3em}  % prevent overfull lines
\providecommand{\tightlist}{%
  \setlength{\itemsep}{0pt}\setlength{\parskip}{0pt}}
\setcounter{secnumdepth}{0}
% Redefines (sub)paragraphs to behave more like sections
\ifx\paragraph\undefined\else
\let\oldparagraph\paragraph
\renewcommand{\paragraph}[1]{\oldparagraph{#1}\mbox{}}
\fi
\ifx\subparagraph\undefined\else
\let\oldsubparagraph\subparagraph
\renewcommand{\subparagraph}[1]{\oldsubparagraph{#1}\mbox{}}
\fi

%%% Use protect on footnotes to avoid problems with footnotes in titles
\let\rmarkdownfootnote\footnote%
\def\footnote{\protect\rmarkdownfootnote}

%%% Change title format to be more compact
\usepackage{titling}

% Create subtitle command for use in maketitle
\newcommand{\subtitle}[1]{
  \posttitle{
    \begin{center}\large#1\end{center}
    }
}

\setlength{\droptitle}{-2em}

  \title{Complete}
    \pretitle{\vspace{\droptitle}\centering\huge}
  \posttitle{\par}
    \author{Amieroh Abrahams}
    \preauthor{\centering\large\emph}
  \postauthor{\par}
    \date{}
    \predate{}\postdate{}
  

\begin{document}
\maketitle

{
\setcounter{tocdepth}{4}
\tableofcontents
}
\begin{Shaded}
\begin{Highlighting}[]
\KeywordTok{print}\NormalTok{(}\KeywordTok{getwd}\NormalTok{())}
\end{Highlighting}
\end{Shaded}

\begin{verbatim}
## [1] "/Users/ajsmit/Dropbox/R/students/Amieroh/Project_writeup"
\end{verbatim}

\newpage

\hypertarget{plagiarism-declaration}{%
\subsection{Plagiarism Declaration}\label{plagiarism-declaration}}

\newpage

\hypertarget{declaration}{%
\subsection{Declaration}\label{declaration}}

\newpage

\hypertarget{abstract}{%
\subsection{Abstract}\label{abstract}}

The South African coastline is comprised of distinct coastal regions,
each varying in temperature but more importantly temperature
characteristics vary among sites. Seawater temperature is an important
driver affecting marine biodiversity and so understanding the factors
affecting the variations of this is important. Here, I analysed
temperature time series data from 18 different sites distributed along
the South African coastline. These data were collected using underwater
temperature recorders (UTRs), alcohol thermometers, and satellites, with
the precision of the instruments varying from between 0.5ºC to 0.001ºC.
Using the minimum, maximum and mean temperatures I systematically
grouped sites together in their distinct coastal regions and compared
temperatures between regions. I further assessed the effects of wind and
wave exposure on the variations of temperatures using coefficient of
determination analyses. My results showed that wind and wave action per
site were not significantly affecting temperature, achieving low R2
values, indicating that wave and wind action were not the predominant
factors influencing coastal seawater temperature. Large scale oceanic
processes such as coastal upwelling and the presence of major ocean
currents, solar radiation or lower atmospheric temperature as well as
anthropogenically induced factors are predicted to be the main drivers
affecting temperature. This suggests that human interference may be
indirectly influencing marine biodiversity via affecting ocean
temperatures, and this insight could prove useful in aiding in
conservation.

\emph{Keywords}: Seawater temperature, climate change, coastal regions,
code: R, variability

\newpage

\hypertarget{acknowledgments}{%
\subsection{Acknowledgments}\label{acknowledgments}}

Special thanks go to my supervisor, Prof.~AJ Smit, for the continuous
motivation and guidance throughout my honours year. All of your
contributions and efforts towards my academic career and experiences
have been appreciated beyond measure. I would also like to thank
Dr.~Robert Schlegel and my co-supervisor, Dr.~Robert Williamson for the
help with my R analyses. I would like to thank KZNSB, DEA, DAFF, SAEON,
SAWS, UWC and EKZNW for contributing the raw data allowing me to do this
study. Without the data this thesis would not be possible.I would also
like to thank my colleagues within Team Kelp who assisted me throughout
this project. Their discussions and friendship has largely aided in the
completion of this thesis. To my friends and family, thank you for
providing me with the necessary support and encouragement. I would like
to thank the entire Biodiversity and Conservation biology department at
the University of the Western Cape. Lastly, I would like to thank the
NRF for providing the necessary funding towards the completion of this
project.

\newpage

\hypertarget{introduction}{%
\section{Introduction}\label{introduction}}

Seawater temperature is a key indicator of environmental change in
marine ecosystems (Pearce, Faskel, and Hyndes 2006), and yet little is
known regarding the controlling influences of temperatures within
coastal zones (with the coastal zone here defined as the region ≤ 400m
from the shore). Coastal temperature observations are generally limited
in their spatial and temporal coverage. Oceanic regions however are
studied to great extent due to availability of long term datasets from
moorings and satellites of monitoring ocean surface temperatures
(Rouault et al. 2010; Beal et al. 2011; Tapia et al. 2014; Lee et al.
2018). Nearshore processes, such as wave action, coastal winds, and
surface radiant heating, and the thermal properties of the substratum,
are a few of the factors that have been implicated in affecting thermal
variability across small spatial scales (Woodson et al. 2007; Davis et
al. 2011; Fewings and Lentz 2011; Sinnett and Feddersen 2014). Given the
significance of the temperature variation for the biogeographical limits
of organisms, due to its effects on the reproductive, growth and
survival limits of species (Hoek 1982; Breeman 1988; Pearce, Faskel, and
Hyndes 2006; Broitman et al. 2008; Byrne et al. 2009; Smale and Wernberg
2009; Smit et al. 2013), it is imperative to understand how marine
organisms may respond to climatic variation in coastal regions on both a
global and local scale. Developing an understanding of the physical
variables present within the coastal zone that are able to mediate
thermal patterns and processes across small spatial scales and short
temporal scales that are typically associated with nearshore processes
will be instrumental in this understanding.

Temperature variability of the coastal region of South Africa, spanning
approximately 3,100 km in distance (Smit et al. 2013), has not yet been
studied in great detail at highly localised scales. At the broad scale,
this region exhibits a large variation in seawater temperatures along
its coastline (Mead et al. 2013; Smit et al. 2013) and is divided into
four bioregions, each with contrasting temperatures. These bioregions
are the Benguela Marrine Province (BMP), Benguela-Agulhas Transition
Zone (B-ATZ), the Agulhas Marine Province (AMP) and the East Coast
Transition Zone (ECTZ) (Smit, Bolton, and Anderson 2017). These regions
display noticeable differences in seawater temperatures in comparison to
each other, primarily due to the influences of the neighbouring ocean
currents (Bolton et al. 2004; Mead et al. 2013; Schlegel and Smit 2016).
These temperature gradients are associated with differences in ecosystem
physiology, species distribution, and habitat structure (Smale and
Wernberg 2009; Wernberg et al. 2010, 2011; Smit, Bolton, and Anderson
2017). As a result of the diverse habitats defined by thermal
differences and exposure gradients along the coastline, species
diversity is not uniformly distributed; consequently, the east and south
coast has much higher species diversity and beta-diversity compared to
the west coast (Mead et al. 2013; Smit, Bolton, and Anderson 2017).

On broad scales, the influences due to the the Benguela and Agulhas
Currents greatly affect the thermal climatologies of the nearshore in
the west and the east of the subcontinent, respectively. At an even
broader scale, the Agulhas Current is driven by a wind stress curl
between the southeast trade winds and the Southern Hemisphere westerlies
(Beal et al. 2011), while the Benguela Current {[}AJS: add the
broader-scale drivers of the Benguela here\ldots{}{]}. Regionally, the
Benguela Current assists in transporting cold water northwards from the
Southern Ocean to the coast (Lüning 1990; Lutjeharms, Cooper, and
Roberts 2000; Hutchings et al. 2009; Schlegel et al. 2017), whereas the
Agulhas Current transports sub-tropical, warm water towards the tip of
Africa (Schlegel et al. 2017). Together these two currents are
responsible for the presence of a strong west-east thermal gradient
occurring along the coastline of South Africa, with the west coast
having significantly colder waters than the east coast (Smit et al.
2013; Smit, Bolton, and Anderson 2017). The south coast is unique as it
is affected by both the Benguela and Agulhas Currents, with a strong
overlap region from Cape Agulhas to Cape Point (Smit, Bolton, and
Anderson 2017), and it experiences a greater spatial and temporal
variation in temperature compared to elsewhere along the coast
(Lutjeharms and Van Ballegooyen 1988). At the localised scale, the
statistical properties of temperature climatologies, such as the mean,
minimum, and maximum of \emph{in situ} coastal seawater temperature time
series for the South African coastline, show distinct coastal variations
(Schlegel and Smit 2016). The local influences acting on the water
masses originating from the Benguela and Agulhas Currents can introduce
thermal variation of up to 10°C within a 24-hour period (Schlegel et al.
2017), thus creating a highly dynamic nearshore environment.

Climate change is often understood as a long-term rise in the global
mean temperatures and has resulted in an increased mean ocean
temperatures over the past few decades (Stocker 2014). The seawater
temperatures of the Benguela Current has been decreasing at a rate of
approximately 0.5°C per decade whilst the Agulhas Current has been
increasing by between 0.55°C-0.7°C per decade (Rouault, Penven, and Pohl
2009; Rouault et al. 2010). Overall, sea surface temperatures (SST)
around South Africa have increased by approximately 0.25°C between 1903
and 2013 (DEA, 2013) and are still increasing at a rate of 0.12 °C per
decade (Schlegel et al. 2017). Climate change is also leading to an
increase in extreme atmosphetic heating (Easterling et al. 2000; Perkins
and Alexander 2013) and a decrease in extreme cold events (Meehl and
Tebaldi 2004). Human activities are largely responsible for these
decadal trends (Rouault, Penven, and Pohl 2009; Rouault et al. 2010 ;
Mead 2011; Mead et al. 2013).

Over the last few decades, improvements in remote sensing technology
have enabled researchers to map global sea surface temperature with a
high level of accuracy (Zainuddin et al. 2006; Smale and Wernberg 2009).
The National Oceanic and Atmospheric Administration's (NOAA) series of
satellites have provided global SST datasets from the 1980s on both
global and local scales (Pearce, Faskel, and Hyndes 2006). The NOAA
dataset is critically important as it is often used to monitor changes
in oceanic temperatures, and provide valuable information on both
biological and physical parameters in the ocean (Demarcq et al.~2010).
Furthermore, satellite-derived SST data play an important role in
creating projections of the potential effects of climate change on
coastal and oceanic marine biota (Müller et al. 2009; Wethey et al.
2011; Bartsch, Wiencke, and Laepple 2012). Satellite-derived data are
not as reliable as \emph{in situ} temperature measurements when used
near the shoreline (Smit et al. 2013), but are often used as a proxy
when these measurements are scarce or unavailable (Smale and Wernberg
2009). However, in South Africa, the local availability of an \emph{in
situ} collected coastal temperature data product provides a reliable
source of accurate coastal seawater temperature data (Smit et al. 2013).
The South African Coastal Temperature Network (SACTN) has collected SST
data form the South African coastline from as early as 1972, with
contributions from various organisations and governmental departments.
This data set, used in combination with satellite-derived data that give
a broader view, provides an opportunity to launch an investigation into
the mechanistic underpinning for why the thermal \emph{milieu} of the
nearshore environment is so dynamic across short time scales and over
short distances along the shore.

The intention of this study is to examine variations in temperature
between selected sites along the South African coastline using seawater
temperature data to better understand patterns of coastal temperature at
a localised scale. The SACTN dataset used in this study consisted of
\emph{in situ} coastal seawater temperature measurements, allowing for
comparisions between sites at a high temporal frequency. We also use
co-located and overlapping satellite datasets, including that of SST,
winds, and waves, to provide measurements of influential variables
(i.e.~as hypothesised drivers of the nearshore temperature field)
representative of the wider regional scale. The aims of this study are
to: i) examine whether there is homogeneity between the various sites
sampled; ii) examine whether or not wind and wave action may contribute
towards a variation in seawater temperatures along the South African
coastline; and iii) to examine whether or not SST data collected via
satellite may be affected by wind and wave action.

\hypertarget{methods}{%
\section{Methods}\label{methods}}

In order to compare abiotic variables such as wind and wave action along
the South African coastline, large historical datasets for temperature,
wave and wind were analysed and accessed.

\hypertarget{wave-and-wind-data}{%
\subsection{Wave and wind data}\label{wave-and-wind-data}}

Wind and wave action were important variables in this study as they were
hypothesised to exhibit a direct influence on coastal water temperatures
(Sinnet and Feddersen, 2014); consequently, they were investigated for
their impact on seawater temperature at specific sites along the South
African coastline. Wind and wave data were obtained from the South
African Weather Service (SAWS), and were provided at three hour
resolutions. Specific wind and wave characteristics were measured,
namely, wave height (hs), wave period (tp), wave direction (dir), wind
direction (dirw) and wind speed (spw). The data were then used to model
short--crested waves, generated by the wind into the coastal
environment, using the wave model Simulating Waves in the Nearshore
(SWAN). SWAN enables the extraction of wave parameters from specific
gridded locations in the nearshore. A resolution of 200 meters was
modelled at both 7 and 15m contours.

\hypertarget{in-situ-emperature-data}{%
\subsection{\texorpdfstring{\emph{In situ} emperature
data}{In situ emperature data}}\label{in-situ-emperature-data}}

The SACTN dataset was the primary source of temperature data used in
this study. This dataset consisted of coastal seawater temperatures for
129 sites along the coast of South Africa, measured daily from 1972
until 2017. Of these, 80 were measured using hand-held thermometers and
the remaining 45 were measured using UTRs. The duration and extent of
the recordings per site were uneven, with the longest time series in the
dataset being that of Gordons Bay, recorded by SAWS. Data collected for
this region started on 13 September 1972 and concluded on 26 January
2017, with recordings still continuing daily. During the 1970s, a total
of 11 time series began recording. A further 53 entries were added
during the 1980s, 34 entries were added during the 1990s, and 18 entries
were added during the 2000s. Recordings are still ongoing at many of
these sites.

For this analyses, the data were combined and formatted into
standardized comma delineated values (CSV) files which allowed for a
fixed methodology to be used across the entire dataset. Prior to data
analysis, all data points exceeding 35 °C and/or below 0 °C were removed
as these were considered as outliers. These data points were then
changed to NA (not available) so as to not interfere with analysis. All
analyses were conducted in R software version 3.4.2 (insert the
reference!!!). The data used within this study and comprehensive script
used for data analyses, and production of figures can be found at
\url{https://github.com/AmierohAbrahams/HONOURSPROJECT}.

\hypertarget{satellite-sst-data}{%
\subsection{Satellite SST data}\label{satellite-sst-data}}

This study made use of four satellite-derived SST datasets to compare
with the SACTN \emph{in situ} coastal seawater temperature and wave
datasets. The AVHRR-only Optimally-Interpolated Sea Surface Temperature
(OISST) was used to determine SST within the study region. The AVHRR
datasets have been provided global SSTs for more than four decades
(Reynolds and Smith, 1994; Pearce et al 2006). OISST is a global 1/4°
gridded daily SST product that assimilates both remotely sensed and
\emph{in situ} sources of data to create a level-4 gap free product
(Banzon et al., 2016). The Multi-scale Ultra-high Resolution (MUR) Sea
Surface Temperature Analysis, the second dataset, is produced using
satellite instruments with datasets spanning 1 June 2002 to present
times (refs.). MUR provides SST data at a spatial resolution of 0.01° in
longitude-latitude coordinates and is currently among the highest
resolution SST datasets available. The third dataset, K10, is produced
at the Naval Oceanographic Office (NAVOCEANO) on a 10km resolution,
globally (refs.). The K10 analysis makes use of SST observations from
the AVHRR, the Geostationary Operational Environmental Satellite (GOES)
Imager and the Advanced Microwave Scanning Radiometer for EOS (AMSR-E).
The CMC dataset constitutes the forth dataset and is a version 3.0 Group
for High Resolution Sea Surface Temperature (GHRSST) Level 4 dataset
with a 10km resolution constructed by the Canadian Meteorological Center
(CMC; refs.). The CMC dataset combines infrared satellite SST at
numerous points in the time series from the AVHRR, the European
Meteorological Operational-A (METOP-A) and Operational-B (METOP-B)
platforms, and microwave SST data from the Advanced Microwave Scanning
Radiometer 2 in conjunction with \emph{in situ} observations of SST from
ships and buoys from the ICOADS program.

\hypertarget{site-selection}{%
\subsection{Site selection}\label{site-selection}}

In order to compare temperatures along the South African coastline at a
localised scale, we selected appropriate sites from each of the major
coastal regions. Since temperature data were not evenly recorded for
each of the 129 sites representing South Africa's coastline, we firstly
narrowed down the full dataset to only those sites that could be
adequately compared. To do this a clustering analysis was perfomed using
the \texttt{kmeans()} function in R, with multiple random seeds to
identify a number of clustering solutions that grouped sites together
based on their available temperature data. The mean, minimum, and
maximum temperature values were used within the clustering algorithm to
group sites with similar temperatures along each coastal region. The
clustering analysis represented the most accurate and distinct site
groupings based on temperature distributions and yielded distinct east,
south, and west coast groupings. Eight clustering solutions along the
South African coastline are shown (Figure 1). With the data now divided
into eight distinct coastal regions, portions of overlapping time series
(\emph{i.e.} across the multiple sites per region) were selected of at
least one decade in duration, but excluding those sites with temperature
data collected deeper than 5m.

\begin{figure}
\centering
\includegraphics{../figures/map_fixed.pdf}
\caption{A map of the study area representing the 129 sites where
\emph{in situ} coastal seawater temperature was collected. These sites
were grouped based on similar mean, minimum and maximum temperatures and
as such each groups represents a unique colour variation.}
\end{figure}

Once sites were clustered, we reduced the number of sites to a
manageable but still representative sub-sample of the whole. This was
done for two reasons. The first was to allow for the comparisons to be
more readily interpretable by humans. Secondly, it was to allow for
equal amount of sampling per coast. The east coast has previously been
more heavily sampled than the rest, and such an imbalance needed to be
addressed. The criteria considered for the sub-samples included
selecting the longest time series within the region and including data
from as many different sources (\emph{i.e.} contributors to the SACTN)
as possible. This process yielded three sites for each of the clusters
along the South African coastline (Figure 2). The statistical
characteristics of the temperature were used to guide analysis of the
time series to produce an accurate assessment of temperature variation
between sites that were grouped together.

\begin{figure}
\centering
\includegraphics{../figures/final_combined_plot.pdf}
\caption{A map representing the three sites chosen within each of the
six cluster along the South African coastline.}
\end{figure}

With three sites per cluster, the haversine formula was used to
calculate the geodesic distance between points specified by radians.
Using this formula, the distances (km) between each of the sites along
the coastline were determined. Thereafter, sites within the same cluster
were matched based on the date that temperature was collected. This
allowed for a comparison to test whether or not temperature variation
exists between sites within the same cluster. Once the sites were
matched, the means and standard deviations of temperatures between sites
and clusters were determined. This highlighted the temperature variation
between matched sites and allowed for seasonal comparisons within the
same cluster.

Once the temperature variation between sites were carefully analysed,
the seawater temperature data along with the wave and wind data were
compared. The data were modelled for water depths of 7m and 15m. Since
the wave and wind data were modelled at three hour resolutions, they
were converted into daily data points in order to compare them with the
temperature data. The \texttt{circular()} function in R software was
then used to create circular objects around the wave data in order to
calculate the daily wave and wind parameters.

With temperature and wave values now corresponding to their respective
sites, depths and dates, the hypothesis regarding whether or not a
relationship existed between wind/wave action and temperature was
tested. To do this, linear models for each site were produced,
reflecting temperature and wave variations at each depth. Linear models
typically produce coefficients of determination
\emph{R}\textsuperscript{2} as an output. The \texttt{purrr()} function
within the \textbf{tidyverse} R package was used to simultaneously
compare temperature and wave data across sites and depths. An ANOVA
analyses was done compare one variable in two or more groups taking into
account the variability of other variables. Hereafter, a wind rose
diagram was constructed to determine the most predominant direction for
a particular site. This was done to ascertain what the potential
relationship between wind/waves and temperature was at each site during
only the prevailing wind directions.

\hypertarget{satellite-data-analysis}{%
\subsection{Satellite data analysis}\label{satellite-data-analysis}}

SST measurements used in this study were obtained from four different
sources: MUR, CMC, K10 and AVHRR. A time series of SST was determined by
creating a bounding box which represented the region of extent at the
latitudes (39.5°S, 25.5°S), and the longitudes (10.5°E, 39.5°E). The
size of the pixel search area was set to a 5km resolution from each of
the stations. The satellite datasets and the corresponding SACTN
\emph{in situ} collected dataset were matched based on the coordinates
and the date at which temperature was collected. Some sites, however,
shared the same satellite data due to their close proximities. Once the
satellite data corresponded to the \emph{in situ} collected data, linear
models for each site were produced, reflecting temperature and wave
variations at each depth. Linear models typically produce coefficients
of determination (\emph{R}\textsuperscript{2} values) as an output,
which is the statistic showing how much of the variance in a dependent
variable is explained by the independent variable. This allows for us to
test whether or not wave and wind direction may influence temperature at
the various sites.

\hypertarget{statistical-analyses}{%
\subsection{Statistical analyses}\label{statistical-analyses}}

A series of ANOVA tests were used to compare the main effects of the
chosen variables on a continuous variable. In this analysis the
relationship between matched sites based on the mean temperature as a
function of year and season were analysed; these analyses tested if
significant differences occured between each pair of sites within each
of the clusters. To determine the strength of correlation of temperature
between sites found within the same clusters along the coast, boxplots
were constructed. These plots enabled the visual identification of
variations in temperature by summarising the descriptice statistics. To
furthur analyse the temperature variation between sites line graphs were
constucted. This plot allowed for visual identification of the variation
in average temperature for each of the month and year for paired sites.

\hypertarget{results}{%
\subsection{Results}\label{results}}

\hypertarget{temperature-variation}{%
\subsubsection{Temperature variation}\label{temperature-variation}}

Seawater temperature was not uniformly distributed across the six
clusters produced (Figure 3), with each set of sites having unique
patterns of temperature variation. Within Cluster 1, along the south and
east coast, comprising of Hamburg, Eastern Beach and Orient Beach,
temperature varied from approximately 13ºC to 22ºC. Within this cluster
of sites, Hamburg had the highest maximum temperatures and the lowest
minimum temperatures of the three sites. Conversely, Orient Beach had
the lowest range of temperature. Orient Beach and Eastern Beach had
relatively similar ranges and distributions of temperatures.

Along the south coast, within the cluster comprised of Mossel Bay,
Stilbaai and Knysna, temperatures ranged from approximately 12ºC to
27ºC. Stilbaai had the widest range of temperature variation among the
three sites, but despite the apparent differences in temperature ranges
between these sites, the average temperatures were relatively similar.
Average temperatures were nearly identical within this cluster, with
very few outliers present within the temperatures ranges of these sites.

Sites located within the third cluster had slightly lower temperatures
than the previous two clusters. This cluster comprised of Bordjies,
Saldanha, and Gansbaai and temperatures within here ranged from
approximately 11ºC to 21ºC, with a median temperature being close to
15ºC across all three sites. Gansbaai had relatively low variation in
temperature. Conversely, Saldanha had a high variation and relatively
evenly distributed temperatures by showing little skewness. These sites
were similar in terms of their temperature. There were however, several
outliers present within the temperatures of these sites.

The fourth cluster which was located along the east coast, comprised of
Port Edward, Leisure Bay, and T.O. Strand. Overall, the temperatures of
these sites were higher than those of the sites within the other
clusters, with a range of 15ºC to 25ºC. These sites dispalyed a low
variation in temperature, with little skewness across sites.
Temperatures were identical between these three sites. The median
temperature for each of the sites within this cluster is 20.5ºC.

Sites within the fifth cluster had overall lower temperatures than those
within the other clusters. This cluster comprised Port Nolloth, Lamberts
Bay, and Sea Point; here sharp declines in average temperatures were
observed throughout. Temperatures within this cluster ranged between 8ºC
and 18ºC, with an average temperature being close to 13ºC. Port Nolloth
had low variation in temperature. Lamberts Bay and Sea Point were
similar in terms of temperature variances. Several outliers were present
within the temperatures of these sites.

In the cluster comprising of Kalk Bay, Muizenberg, and Gordons Bay
temperatures ranged from 8ºC to 24ºC. Muizenberg had the widest range of
temperature variation of the three sites. Gordons Bay and Kalk Bay had
identical temperature ranges. Similarly to the second cluster, despite
the apparent differences in temperature ranges between these sites, the
median temperatures across them were relatively similar and nearly
identical.

\begin{figure}
\centering
\includegraphics{../figures/combined_plot.pdf}
\caption{Boxplots representing the seawater temperature of paired sites
within each cluster along the South African coast with the values
representing the distance (km) between paired sites.}
\end{figure}

On a monthly basis, large differences of average temperatures were seen
between sites within Cluster 1. These differences largely occured during
the summer and spring months of 1995 to 1997 (Figure 4). For the
remaining sites, however, differences in average temperatures were lower
during autumn. It was also evident that the average temperature between
Hamburg and Orient Beach {[}AJS: and also Eastern and Orient?; You might
also say that the Hamberg / Eastern comparison is not reliable due to
the short amount of overlap, and hence it doesn't quite stabilise the
longterm trends.{]} varied largely on an apparent seasonal basis. Small
monthly average temperature differences existed between Eastern Beach
and Orient Beach throughout the different seasons.

Converse to Cluster 1 {[}AJS: call them by their names, and use that
rather than listing all the sites per cluster.{]}, the sites within
Cluster 2 displayed the largest differences in average temperatures
during spring. In this cluster, large differences in average
temperatures were present between Mossel Bay and Knysna, with the
differences increasing annually from 1985 to 2017. Similarly,
differences in average temperature also increased slightly between
Stilbaai and Knysna during winter and spring. During summer months
little differences in average temperature were seen between all three
sites within this cluster.

Differences of average temperatures between sites within Cluster 3
varied on a seasonal basis. During the summer months, large differences
in average temperatures existed between Bordjies and the remaining two
sites, with an increase in differences of average temperature between
Saldanha Bay and Gansbaai throughout autumn, winter and spring.

In Cluster 4, small changes in the pairwise differences of monthly
average temperatures were noticed between sites between 1980 to 2017.
Here, large differences in temperatures were observed towards the end of
spring and during summer months. Temperatures were relatively stable
throughout winter and the beginning of spring across all three sites
within this cluster.

In Cluster 5, large differences in average temperatures existed between
sites at selected months between 1972 and 2017. During summer and autumn
months, differences in average temperature were observed between
Lamberts Bay and the remaining sites increased. During the months of
autumn Lamberts Bay and Sea Point showed large differences in average
temperature variation. For the remaining sites, differences in average
temperatures were relatively low throughout each month for the same time
period.

In Cluster 6, the largest differences in average temperatures were
observed during mid-autumn and winter months. In this cluster, large
differences in average temperatures were seen between Muizenberg and the
remaining sites, with the differences increasing annually throughout
1972 and 2016 during winter. Similarly, differences of average
temperatures also increased between Kalk Bay and Muizenberg during these
same months. In the summer and spring months little differences in
average temperatures between sites, with minimal differences in the
rates of these changes. These rates increased during spring.

\begin{figure}
\centering
\includegraphics{../figures/combined2_plot.pdf}
\caption{Line graph representing the pairwise differences in median
seawater temperature, shown individually for each month across years.
The values appended to the end of the legend keys represent the
distances (km) between the paired sites.}
\end{figure}

\hypertarget{average-temperature-of-clustered-sites}{%
\subsubsection{Average temperature of clustered
sites}\label{average-temperature-of-clustered-sites}}

In Cluster 1, the results of a three-way ANOVA {[}AJS: these were
three-way ANOVAs; you must also mention any interaction effects. Maybe
create a table of ANOVA results (see papers for exampels) and mention
only the significant differences here in the text.{]} showed that there
was a significant (\emph{p}\textless{}0.05) difference in average
temperatures between paired sites (\emph{F}=12.07, \emph{SS}=15.28,
\emph{p}\textless{}0.001). These differences were present across season
(\emph{F}=3.44, \emph{SS}=13.07, \emph{p}\textless{}0.002) but were not
present yearly between individual sites and paired sites (\emph{F}=1.38,
\emph{SS}=1.75, \emph{p}=0.25). Similarly, paired sites within Cluster 2
also displayed a significantly difference in average temperature
(\emph{F}=166.84, \emph{SS}=418.6, \emph{p}\textless{}0.001). Conversely
to Cluster 1, however, these differences were present yearly
(\emph{F}=33.21, \emph{SS}=41.7, \emph{p}\textless{}0.001) and
seasonally (\emph{F}=16.72, \emph{SS}=125.9, \emph{p}\textless{}0.02)
between individual and paired sites. In Cluster 3, a three-way ANOVA
revealed no significant differences in average temperatures between
paired sites (\emph{F}=1.17, \emph{SS}=2.9, \emph{p}=0.31), but
temperatures varied seasonally and yearly between individual sites. In
Cluster 4 there were again no significant difference in average
temperatures between paired sites (\emph{F}= 0.73, \emph{SS}=2.9,
\emph{p}=0.48). Significant differences were also absent across seasons
(\emph{F}=0.75, \emph{SS}=0.0042, \emph{p}=0.52) and years
(\emph{F}=0.495, \emph{SS}=0.0009, \emph{p}=0.48) between both
individual and paired sites. Sites within Cluster 5 were significantly
different in average temperatures between paired sites (\emph{F}=77.10,
\emph{SS}=196.7, \emph{p}\textless{}0.002), with these differences being
present yearly (\emph{F}=172.80, \emph{SS}=220.4,
\emph{p}\textless{}0.001) and seasonally (\emph{F}=77.10,
\emph{SS}=29.92, \emph{p}\textless{}0.002) for both paired and
individual sites. Finally, Cluster 6 sites also had significant
difference in average temperatures between paired sites, both yearly and
seasonally (\emph{F}=132.044, \emph{SS}=419.7, \emph{p}\textless{}0.01).

\hypertarget{the-impact-of-wind-and-wave-action-on-temperature}{%
\subsection{The impact of wind and wave action on
temperature}\label{the-impact-of-wind-and-wave-action-on-temperature}}

The coeffcient of determination (\emph{R}\textsuperscript{2}) was
calculated to determine how differences in one variable can be explained
by a difference in the second variable. Here, the relationship between
temperature and several variables, including wind direction and speed as
well as wave height, period, direction and speed, was determined for all
18 sites. The findings revealed little indluence of the hypothesised
predictor variables on \emph{in situ} SACTN temperatures for each of the
sites. Wind and wave direction influenced temperature at 0--3\%. The
most significant {[}AJS: What do you mean with `most significant'? Was
the p-values \textless{}0.05?{]} relationships were found in Muizenberg,
Kalk Bay and Mossel Bay where the \emph{R}\textsuperscript{2} values
indicated that wave period had a 6-7\% influence on temperature {[}AJS:
but is this significant?! You need to also mention the p-values and
d.f., etc.{]}. Overall wind and wave action had no significant impact on
temperature differences along the coast.

Upon examining the impact of wind and wave action on seawater
temperature of the AVHRR temperature dataset, it emerged that wave and
wind direction had a minimal effect on temperature variation in this SST
dataset, with its \emph{R}\textsuperscript{2} values ranging between
0-3\%. The results did however suggest that wave height might have
minimally influenced temperature at some of the sites, namely at
Gansbaai and Lamberts Bay where wave height had a 9\% {[}AJS: again
p-values etc. must be given{]} impact on temperature variation. The
results obtained from the MUR dataset continued to show little variation
in regards to the influence of wind and wave action on temperature. At
many of the sites, both wind and wave direction had a 0\% impact on the
temperature; however, it was seen that wave height and wave period had
the greatest impact on temperature at some of the sites, with Gordons
Bay and Gansbaai indicating an 8\% and 10\% impact of wave height on
temperature variation, respectively. The results obtained from the CMC
temperature dataset indicated that wave and wind direction as well as
wind speed showed the least significant impact on temperature variation
with \emph{R}\textsuperscript{2} values ranging between 0-3\%. Wave
height continued to show the largest impact on temperature variation at
some of the sites. Gansbaai and Gordons Bay indicated the highest
\emph{R}\textsuperscript{2} values of 12\% and 9\% respectively. Upon
comparing the impact of various environmental factors on temperature
variation for the K10 data, the results indicated that the each of the
above-mentioned variables had no impact on the temperature variation at
Hamburg. The impact of wind direction on temperature is highest at
Gansbaai, representing an \emph{R}\textsuperscript{2} value of 4\% while
wave height still repersented the greatest influence on temperature
occurring at these sites.

\hypertarget{wind-rose-diagram}{%
\subsection{Wind rose diagram}\label{wind-rose-diagram}}

{[}AJS: why not create a similar diagramme for the waves?{]}

Wind and wave diagrams help visualise the patterns present at a
particular site. Moving outward on the radial scale, the frequency
associated with wind and waves coming from a partiular direction
increases. The predominant wind direction along the south coast 105º.
The predominant wind direction of sites located along the east coast
such as T.O. Strand and Orient Beach occured at 45º. Leisure Bay, also
located along the east coast however indicates a predominant wind
direction of 15º. Port Nolloth, located along the west coast, indicates
a predominant wind direction at both 135º and 165º.

\begin{figure}
\centering
\includegraphics{../figures/p.wr3.pdf}
\caption{Wind rose diagrams representing the the most predominant wind
direction for each of the sites. Each spoke is divided by color into
wind speed ranges. The radial length of each spoke around the circle is
the time that the wind and waves comes from that direction}
\end{figure}

\hypertarget{the-impact-of-predominant-wind-and-wave-action-on-seawater-temperature}{%
\subsection{The impact of predominant wind and wave action on seawater
temperature}\label{the-impact-of-predominant-wind-and-wave-action-on-seawater-temperature}}

The results indicated that wind speed, wind height, wind and wave
direction, as well as wave period had no significant impact on \emph{in
situ} temperature variation at the various sites along the coastline
(Figure 6). Eastern Beach, however, showed that wave period and wave
height appear to have the largest effect on temperature variation. At
Hamburg and Gordons Bay, wave direction explained 4\% of the temperature
variation. At Muizenberg the largest influence of 5\% was due to wave
height on temperature variation. At Cluster 2, wind and wave action also
had very little impact on temperature variation. In Cluster {[}AJS:
X{]}, wave height was seen to have had a minor impact on temperature
variation with an \emph{R}\textsuperscript{2} value varying between
1-5\%. Overall, it is seen that predominant wind and wave direction have
no significant impact on the seawater temperature variation along the
coast. {[}AJS: again, show the associated p-values.{]}

\begin{figure}
\centering
\includegraphics{../figures/predominant_ww.pdf}
\caption{Boxplot representing the \emph{R}\textsuperscript{2} value each
of the environmental variables at the most predominant wave and wind
direction for each of the sites.}
\end{figure}

\hypertarget{discussion}{%
\subsection{Discussion}\label{discussion}}

\textcolor{red}{why was it expected that wind and wave may influece tempeature}

This study aimed to investigate how wind and wave action influenced
variation in coastal seawater temperature along the South African
coastline. As seawater temperature is known to have large influences on
species distributions (Bolton 2010; Smit et al. 2013), it is important
to understand not only how temperatures vary along the coastline, but
also the factors driving these variations as well. In this way, the
proximal causes of environmental change might be understood in terms of
their relevance in affecting ecological patterns and processes along the
coastline. Sinnett and Feddesen (2014) showed that various environmental
factors such as solar radiation, air temperature, humidity and wave
energy are responsible for temperature variation within the coastal
region. Here, however, additional influences were examined (\emph{i.e.}
broader-scale SST fields, and prevailing winds and waves impacting the
ocast), but none of these variables were found to offer any mechanistic
influence on the alongshore thermal variability along the South African
shore.

Specificaly, across each of the 18 sites that were assessed, we found
wind and wave action to not significantly influence both satellite and
\emph{in situ} SSTs. Differences in temperature between sites at
different locations along the coastline were therefore not caused by
variations in wind or wave action, suggesting that other factors were
responsible for the observed patterns of temperature variation. These
findings were consistent across six different datasets from various
sources, with each of these providing minimal evidence of a relationship
between wind or wave action and the SST of a site. It must be noted that
these analyses consisted of a combination of satellite and thermometer
data, which varied in their measurements and time series--this
consistency of findings across independent datasets is a strength of
this analysis.

Theron et al.~(2014) discussed how the wave dataset, from which the
current study data were extracted, was created and validated. However,
despite the validation, discrepancies may arise as a result of the
assumptions made in the SWAN model. Quasi-stationary SWAN computation
were performed under the assumption that the boundary conditions were
fluctuating at a much slower tempo compared to the time it took for
those conditions to propagate towards the coastline. This may ultimately
result in the wind driven wave components to be overestimated as the
duration limited effect of the wind was thus neglected and computed
towards a converging wave condition during each quasi-stationary time
step. The SWAN model usually overestimates the energy of developing
waves with low frequencies (long periods) for very short distances from
the shore. This is because wave conditions are simplified by using an
\emph{a priori} wave spectrum (Booij, Ris, and Holthuijsen 1999; Thomas
and Dwarakish 2015). For most modelled areas, these assumptions were
reasonable to make because most of the South African coastline is
exposed and agree to the requirements of these assumptions (Joubert et
al. 2013).

Previous studies suggest that a longer time series containing more data
would have greater accuracy in detecting subtle changes in temperature
differences obtained from variable coastal regions such as the South
African coastline. Schlegel and Smit (2016) suggested that having a time
series of greater than 30 years for sites with a low variance results in
increased ability to detect changes. They showed that high quality time
series datasets have frequent measurements with minimal missing (NA)
values present. Furthermore, low quality datasets (\emph{i.e.} those
with more than 5\% of NA values) have a higher chance of detecting
variation where none exists.

We would be amiss to point out some weaknesses inherent in the SACTN
dataset. Temperature measurement along the South African coastline
started in the 1970s, and has been inconsistent in terms of instruments
used, modifications to styles and methods, calibration, and site
locations (Smit et al. 2013; Schlegel and Smit 2016). Additionally,
older records within some datasets, such as the SACTN dataset, may have
been lost or are unreliable since the metadata for these are
unavailable. These metadata are essential as their absence prevents the
understanding of the influence that the instruments had on temperature
recordings. Some measurements may therefore not represent accurate or
well-validated temperatures due to the instrumentation used; however, we
mitigated these effects by selecting only the most reliable and longest
duration data, which other studies (Williamson et al., in progress) have
suggested are most suitable for the problem at hand.

Data in the SACTN are comprised of multiple sources, acquired by various
means, and efforst are currently underway to homogenise these data sets
(Williamson et al., in progress). The underwater temperature recorders
(UTRs) used to collect data in the SACTN dataset expressed a lower
number of NA values compared to the data collected with hand
thermometers. As such, this may have influenced the overall time series
dataset (Schlegel and Smit 2016). The level of precision at which data
were collected also influenced the length of the time series needed.
Time series in which temperature were collected at a precision of 0.5ºC
may require another 24 months of recordings to precisely detect long
term variation (Schlegel and Smit 2016). The average length of the
thermometer time series component of the SACTN dataset was 346 months
whereas the average length of UTR time series was less than half of
that. With the extent of these differences in length being so severe,
even once correcting for potential negative effects on the measurement
precision of the thermometer collected time series, it was clear that
thermometer data were more useful than that of UTRs. These influences
may be expected to affect trend estimates of change derived from the
data, but we do not anticipate that they will affect estimates of the
variability in time and between sites in the way that they were used in
this current study.

Satellite-acquired SST records are useful to modern marine scientists.
These data are often used to model and predict a wide range of oceanic
and biological processes in the open ocean, but have only recently been
used to study temperature variations influencing benthic organisms
(Pearce, Faskel, and Hyndes 2006). Here, both satellite SST data and
\emph{in situ} thermometer data were correlated with wind and wave data
along the coastal environment above a depth of 7 and 15m (representing
variable distances from the shore, depending on the local bathemetry).
SST data acquired by satellites deviate from coastal \emph{in situ}
collected seawater temperatures for a variety of reasons (Smit et al.
2013). The SST data presenent the broader-scale situation along the
coast, and we use it here as a forcing boundary that we hypothesise
influence the coastal temperatures. Our analysis shows that neither the
offshore SSTs (various sources) nor the SACTN \emph{in situ} records are
affected by wind and wave action anywhere along the South African
coastline.

Our findings were surprising but not entirely unexpected. Along the
coastline of South Africa, there is a known east-west thermal
temperature gradient that may have caused some interesting influences on
the nature of our findings (Smit et al. 2013). This is caused by major
oceanic processes such as coastal upwelling, thermohaline circulation,
solar radiation, atmospheric temperature ({\textbf{???}}) and the
presence of major ocean currents, which cumulatively, influence the
temperatures along this coastline (Walker 1990; Schlegel and Smit 2016).
While it was reasonable to assume that surface level environmental
factors like wind and wave action would affect sea surface temperatures
at both local and regional scales, it is not completely unexpected that
they would have little effect given the prevalence of the major
processes mentioned above. In other words, the overriding thermal
climatology of the boundary currents imprint their signals on the coast
over the long-term. This study showed that over shorter daily time
scales, SST {[}AJS: does/does not{]} have influneces in finer spatial
scale nearshore temperatures. {[}AJS: discuss the relevance of this now
that you have done the analysis.{]} {[}AJS: I shall revist this once you
have done the above analysis\ldots{}However, this is unlikely as the
`West Coast' system is considered wind driven and there are multiple
wind driven upwelling cells along the south coast. Those processes are
of the largest drivers behind coastal temperature variation and may
simply be overpowering the effects of other environmental factors. Other
factors such as latent heat flux and wave energy flux were also proven
to heat and cool coastal seawater temperatures ({\textbf{???}}).{]}

Given the almost complete lack of support of waves, winds or offshore
temperatures on the nearshore thermal variability, it is likely that
other as yet unknown factors influence temperatures along the South
African coastline. For example, whilst rainfall can have large
influences on coastal SST (Reason and Mulenga 1999) other, non-climatic,
factors could be playing a greater role. Coastal regions are highly
impacted by human mediated pressures (Mead et al. 2013). These pressures
are predicted to drive change over a spatial and temporal scale and is
often a cause of temperature variation (Griffiths, Mead, and Zietsman
2011).

{[}AJS: I don't think these are plausible. I can't think of mechanisms
by which this can work. Think of other things, such as local bathymetry,
vertical mixing, pulsing of thermoclines, stratification due to diurnal
solar heating, internal waves, tides, the effects due to coastal
goemosphology, such as cape/bay effects, etc\ldots{}{]} These pressures
are present along the South African coastline in the form of pollution,
coastal runoff, invasive species, resource exploitation, and coastal
mining, which directly and indirectly influences air temperature, wind,
seawater temperature, and rainfall ({\textbf{???}}; James and Hermes
2011; Mead et al. 2013). These anthropogenic factors may be cumulatively
playing an important role in affecting SST along the coastline, but the
full extent of these factors are unknown.

{[}AJS: not climate change either. Climate change affects long term
trends mostly, and the amplitude and frequency of variability, but I
can't imagine that it influences the degree to which variability across
scales are in sync or not in sync.{]} Additionally, evidence indicates
that human driven climate change on both a local and global scale
largely influence seawater temperatures, wind regimes, and wave action
(Rouault, Penven, and Pohl 2009; Rouault et al. 2010; Mead et al. 2013).

Within marine environments, coastal temperature variation allows for a
variation in the spatial arrangements of marine biodiversity. Whilst
wind and wave action may not be directly affecting ocean temperatures,
Blamey and Branch (2008) have found that wave action has a profound
influence on species distributions along the coastline. The presence or
absence of marine species are determined by a variety of factors and
whilst those factors may not be influencing each other as was the case
here, they collectively play important roles in affecting the marine
life of the South African coastline and identifying those roles can aid
in improving our understanding of nearshore dynamics, thereby providing
greater knowledge to be used for conservation.

This study has shown that wind and wave action are not directly
affecting seawater temperature variation along the South African
coastline, However, other factors may be. Future research could aim to
examine the effects of air temperature and rainfall on the coastal
seawater temperature for the 18 sites being assessed. Additionally,
other factors such as the amount of sunlight penetrating the ocean or
site exposure could also be tested, as well as assessing the optimal
locations for data collection. Further studies should also consider
examining how chlorophyll concentrations and salinity varies with
temperature in order to assess the effects of seawater temperature on
marine plant life. Homogenous coastlines with distinct temperature
gradients such as in South Africa provide a model environment for
temperature analyses at fine-resolutions. These data could provide
critically important information that can be used to assist in
conserving marine biodiversity within these waters and should be given
greater priority within marine research in the near future.

\hypertarget{refs}{}
\leavevmode\hypertarget{ref-Bartsch2012}{}%
Bartsch, Inka, Christian Wiencke, and Thomas Laepple. 2012. ``Global
Seaweed Biogeography Under a Changing Climate: The Prospected Effects of
Temperature.'' In \emph{Seaweed Biology}, 383--406. Springer.

\leavevmode\hypertarget{ref-Beal2011}{}%
Beal, Lisa M, De RuijterWilhelmus PM, Arne Biastoch, Rainer Zahn, Meghan
Cronin, Juliet Hermes, Johann Lutjeharms, et al. 2011. ``On the Role of
the Agulhas System in Ocean Circulation and Climate.'' \emph{Nature} 472
(7344): 429.

\leavevmode\hypertarget{ref-Bolton2004}{}%
Bolton, JJ, Frédérik Leliaert, De ClerckOlivier, RJ Anderson, H
Stegenga, HE Engledow, and Eric Coppejans. 2004. ``Where Is the Western
Limit of the Tropical Indian Ocean Seaweed Flora? An Analysis of
Intertidal Seaweed Biogeography on the East Coast of South Africa.''
\emph{Marine Biology} 144 (1): 51--59.

\leavevmode\hypertarget{ref-Bolton2010}{}%
Bolton, John J. 2010. ``The Biogeography of Kelps (Laminariales,
Phaeophyceae): A Global Analysis with New Insights from Recent Advances
in Molecular Phylogenetics.'' \emph{Helgoland Marine Research} 64 (4):
263.

\leavevmode\hypertarget{ref-Booij1999}{}%
Booij, NRRC, RC Ris, and Leo H Holthuijsen. 1999. ``A Third-Generation
Wave Model for Coastal Regions: 1. Model Description and Validation.''
\emph{Journal of Geophysical Research: Oceans} 104 (C4): 7649--66.

\leavevmode\hypertarget{ref-Breeman1988}{}%
Breeman, AM. 1988. ``Relative Importance of Temperature and Other
Factors in Determining Geographic Boundaries of Seaweeds: Experimental
and Phenological Evidence.'' \emph{Helgoländer Meeresuntersuchungen} 42
(2): 199.

\leavevmode\hypertarget{ref-Broitman2008}{}%
Broitman, BR, CA Blanchette, BA Menge, J Lubchenco, C Krenz, M Foley, PT
Raimondi, D Lohse, and SD Gaines. 2008. ``Spatial and Temporal Patterns
of Invertebrate Recruitment Along the West Coast of the United States.''
\emph{Ecological Monographs} 78 (3): 403--21.

\leavevmode\hypertarget{ref-Byrne2009}{}%
Byrne, Maria, Melanie Ho, Paulina Selvakumaraswamy, Hong D Nguyen, Symon
A Dworjanyn, and Andy R Davis. 2009. ``Temperature, but Not pH,
Compromises Sea Urchin Fertilization and Early Development Under
Near-Future Climate Change Scenarios.'' \emph{Proceedings of the Royal
Society of London B: Biological Sciences} 276 (1663): 1883--8.

\leavevmode\hypertarget{ref-Davis2011}{}%
Davis, KA, SJ Lentz, J Pineda, JT Farrar, VR Starczak, and JH Churchill.
2011. ``Observations of the Thermal Environment on Red Sea Platform
Reefs: A Heat Budget Analysis.'' \emph{Coral Reefs} 30 (1): 25--36.

\leavevmode\hypertarget{ref-Easterling2000}{}%
Easterling, David R, Gerald A Meehl, Camille Parmesan, Stanley A
Changnon, Thomas R Karl, and Linda O Mearns. 2000. ``Climate Extremes:
Observations, Modeling, and Impacts.'' \emph{Science} 289 (5487):
2068--74.

\leavevmode\hypertarget{ref-Fewings2011}{}%
Fewings, Melanie R, and Steven J Lentz. 2011. ``Summertime Cooling of
the Shallow Continental Shelf.'' \emph{Journal of Geophysical Research:
Oceans} 116 (C7).

\leavevmode\hypertarget{ref-Griffiths2011}{}%
Griffiths, CL, A Mead, and L Zietsman. 2011. ``Human Activities as
Drivers of Change on South African Rocky Shores.'' \emph{Observations on
Environmental Change in South Africa, Sun Media, Stellenbosch, South
Africa}, 242--6.

\leavevmode\hypertarget{ref-Hoek1982}{}%
Hoek, C van den. 1982. ``The Distribution of Benthic Marine Algae in
Relation to the Temperature Regulation of Their Life Histories.''
\emph{Biological Journal of the Linnean Society} 18 (2): 81--144.

\leavevmode\hypertarget{ref-Hutchings2009}{}%
Hutchings, L, Van der LingenCD, LJ Shannon, RJM Crawford, HMS Verheye,
CH Bartholomae, Van der PlasAK, et al. 2009. ``The Benguela Current: An
Ecosystem of Four Components.'' \emph{Progress in Oceanography} 83
(1-4): 15--32.

\leavevmode\hypertarget{ref-James2011}{}%
James, Nicola Caroline, and Juliet Hermes. 2011. \emph{Insights into
Impacts of Climate Change on the South African Marine and Coastal
Environment}. SAEON.

\leavevmode\hypertarget{ref-Joubert2013}{}%
Joubert, JR, JL van Niekerk, J Reinecke, and I Meyer. 2013. ``Wave
Energy Converters (Wecs).'' \emph{Centre for Renewable and Sustainable
Energy Studies, Centre for Renewable and Sustainable Energy Studies,
Faculty of Engineering}.

\leavevmode\hypertarget{ref-Lee2018}{}%
Lee, Kate Asha, Moninya Roughan, Hamish Malcolm, and Nicholas Otway.
2018. ``Assessing the Use of Area-and Time-Averaging Based on Known
de-Correlation Scales to Provide Satellite Derived Sea Surface
Temperatures in Coastal Areas.'' \emph{Frontiers in Marine Science} 5:
261.

\leavevmode\hypertarget{ref-Lutjeharms2000}{}%
Lutjeharms, JRE, J Cooper, and M Roberts. 2000. ``Upwelling at the
Inshore Edge of the Agulhas Current.'' \emph{Continental Shelf Research}
20 (7): 737--61.

\leavevmode\hypertarget{ref-Lutjeharms1988}{}%
Lutjeharms, JRE, and Van BallegooyenRC. 1988. ``Anomalous Upstream
Retroflection in the Agulhas Current.'' \emph{Science} 240 (4860): 1770.

\leavevmode\hypertarget{ref-Luning1990}{}%
Lüning, Klaus. 1990. \emph{Seaweeds: Their Environment, Biogeography,
and Ecophysiology}. John Wiley \& Sons.

\leavevmode\hypertarget{ref-Mead2013}{}%
Mead, A, CL Griffiths, GM Branch, CD McQuaid, LK Blamey, JJ Bolton, RJ
Anderson, et al. 2013. ``Human-Mediated Drivers of Change---Impacts on
Coastal Ecosystems and Marine Biota of South Africa.'' \emph{African
Journal of Marine Science} 35 (3): 403--25.

\leavevmode\hypertarget{ref-Mead2011}{}%
Mead, Angela. 2011. ``Climate and Bioinvasives Drivers of Change on
South African Rocky Shores?'' PhD thesis, University of Cape Town.

\leavevmode\hypertarget{ref-Meehl2004}{}%
Meehl, Gerald A, and Claudia Tebaldi. 2004. ``More Intense, More
Frequent, and Longer Lasting Heat Waves in the 21st Century.''
\emph{Science} 305 (5686): 994--97.

\leavevmode\hypertarget{ref-Muller2009}{}%
Müller, Ruth, Thomas Laepple, Inka Bartsch, and Christian Wiencke. 2009.
``Impact of Oceanic Warming on the Distribution of Seaweeds in Polar and
Cold-Temperate Waters.'' \emph{Botanica Marina} 52 (6): 617--38.

\leavevmode\hypertarget{ref-Pearce2006}{}%
Pearce, Alan, Fabienne Faskel, and Glenn Hyndes. 2006. ``Nearshore Sea
Temperature Variability Off Rottnest Island (Western Australia) Derived
from Satellite Data.'' \emph{International Journal of Remote Sensing} 27
(12): 2503--18.

\leavevmode\hypertarget{ref-Perkins2013}{}%
Perkins, SE, and LV Alexander. 2013. ``On the Measurement of Heat
Waves.'' \emph{Journal of Climate} 26 (13): 4500--4517.

\leavevmode\hypertarget{ref-Reason1999}{}%
Reason, CJC, and H Mulenga. 1999. ``Relationships Between South African
Rainfall and Sst Anomalies in the Southwest Indian Ocean.''
\emph{International Journal of Climatology: A Journal of the Royal
Meteorological Society} 19 (15): 1651--73.

\leavevmode\hypertarget{ref-Rouault2010}{}%
Rouault, Marjolaine J, Alexis Mouche, Fabrice Collard, JA Johannessen,
and Bertrand Chapron. 2010. ``Mapping the Agulhas Current from Space: An
Assessment of Asar Surface Current Velocities.'' \emph{Journal of
Geophysical Research: Oceans} 115 (C10).

\leavevmode\hypertarget{ref-Rouault2009}{}%
Rouault, Mathieu, Pierrick Penven, and Benjamin Pohl. 2009. ``Warming in
the Agulhas Current System Since the 1980's.'' \emph{Geophysical
Research Letters} 36 (12).

\leavevmode\hypertarget{ref-Schlegel2017}{}%
Schlegel, Robert W, Eric CJ Oliver, Sarah Perkins-Kirkpatrick, Andries
Kruger, and Albertus J Smit. 2017. ``Predominant Atmospheric and Oceanic
Patterns During Coastal Marine Heatwaves.'' \emph{Frontiers in Marine
Science} 4: 323.

\leavevmode\hypertarget{ref-Schlegel2016}{}%
Schlegel, Robert W, and Albertus J Smit. 2016. ``Climate Change in
Coastal Waters: Time Series Properties Affecting Trend Estimation.''
\emph{Journal of Climate} 29 (24): 9113--24.

\leavevmode\hypertarget{ref-Sinnett2014}{}%
Sinnett, Gregory, and Falk Feddersen. 2014. ``The Surf Zone Heat Budget:
The Effect of Wave Heating.'' \emph{Geophysical Research Letters} 41
(20): 7217--26.

\leavevmode\hypertarget{ref-Smale2009}{}%
Smale, Dan A, and Thomas Wernberg. 2009. ``Satellite-Derived Sst Data as
a Proxy for Water Temperature in Nearshore Benthic Ecology.''
\emph{Marine Ecology Progress Series} 387: 27--37.

\leavevmode\hypertarget{ref-Smit2017}{}%
Smit, Albertus J, John J Bolton, and Robert J Anderson. 2017. ``Seaweeds
in Two Oceans: Beta-Diversity.'' \emph{Frontiers in Marine Science} 4:
404.

\leavevmode\hypertarget{ref-Smit2013}{}%
Smit, Albertus J, Michael Roberts, Robert J Anderson, Francois Dufois,
Sheldon FJ Dudley, Thomas G Bornman, Jennifer Olbers, and John J Bolton.
2013. ``A Coastal Seawater Temperature Dataset for Biogeographical
Studies: Large Biases Between in Situ and Remotely-Sensed Data Sets
Around the Coast of South Africa.'' \emph{PLoS One} 8 (12): e81944.

\leavevmode\hypertarget{ref-Stocker2014}{}%
Stocker, Thomas. 2014. \emph{Climate Change 2013: The Physical Science
Basis: Working Group I Contribution to the Fifth Assessment Report of
the Intergovernmental Panel on Climate Change}. Cambridge University
Press.

\leavevmode\hypertarget{ref-Tapia2014}{}%
Tapia, Fabian J, John L Largier, Manuel Castillo, Evie A Wieters, and
Sergio A Navarrete. 2014. ``Latitudinal Discontinuity in Thermal
Conditions Along the Nearshore of Central-Northern Chile.'' \emph{PLoS
One} 9 (10): e110841.

\leavevmode\hypertarget{ref-Thomas2015}{}%
Thomas, T Justin, and GS Dwarakish. 2015. ``Numerical Wave Modelling--a
Review.'' \emph{Aquatic Procedia} 4: 443--48.

\leavevmode\hypertarget{ref-Walker1990}{}%
Walker, ND. 1990. ``Links Between South African Summer Rainfall and
Temperature Variability of the Agulhas and Benguela Current Systems.''
\emph{Journal of Geophysical Research: Oceans} 95 (C3): 3297--3319.

\leavevmode\hypertarget{ref-Wernberg2011}{}%
Wernberg, Thomas, Bayden D Russell, Pippa J Moore, Scott D Ling, Daniel
A Smale, Alex Campbell, Melinda A Coleman, Peter D Steinberg, Gary A
Kendrick, and Sean D Connell. 2011. ``Impacts of Climate Change in a
Global Hotspot for Temperate Marine Biodiversity and Ocean Warming.''
\emph{Journal of Experimental Marine Biology and Ecology} 400 (1-2):
7--16.

\leavevmode\hypertarget{ref-Wernberg2010}{}%
Wernberg, Thomas, Mads S Thomsen, Fernando Tuya, Gary A Kendrick, Peter
A Staehr, and Benjamin D Toohey. 2010. ``Decreasing Resilience of Kelp
Beds Along a Latitudinal Temperature Gradient: Potential Implications
for a Warmer Future.'' \emph{Ecology Letters} 13 (6): 685--94.

\leavevmode\hypertarget{ref-Wethey2011}{}%
Wethey, David S, Sarah A Woodin, Thomas J Hilbish, Sierra J Jones,
Fernando P Lima, and Pamela M Brannock. 2011. ``Response of Intertidal
Populations to Climate: Effects of Extreme Events Versus Long Term
Change.'' \emph{Journal of Experimental Marine Biology and Ecology} 400
(1-2): 132--44.

\leavevmode\hypertarget{ref-Woodson2007}{}%
Woodson, CB, DI Eerkes-Medrano, A Flores-Morales, MM Foley, SK Henkel, M
Hessing-Lewis, D Jacinto, et al. 2007. ``Local Diurnal Upwelling Driven
by Sea Breezes in Northern Monterey Bay.'' \emph{Continental Shelf
Research} 27 (18): 2289--2302.

\leavevmode\hypertarget{ref-Zainuddin2006}{}%
Zainuddin, Mukti, Hidetada Kiyofuji, Katsuya Saitoh, and Sei-Ichi
Saitoh. 2006. ``Using Multi-Sensor Satellite Remote Sensing and Catch
Data to Detect Ocean Hot Spots for Albacore (Thunnus Alalunga) in the
Northwestern North Pacific.'' \emph{Deep Sea Research Part II: Topical
Studies in Oceanography} 53 (3-4): 419--31.


\end{document}
